\chapter{Introduction}
\label{cha:Intro}

\section{Field of Research}

Fractal geometry is an area of mathematical research that concerns itself with mathematically describing n-dimensional geometries - limited here to geometries on a plane - that display \textit{interesting} characteristics. 

Figure \ref{fig:flosnek} shows a rendering of one such geometry called \textsc{Gosper's flowsnake}.
It is a single, uninterrupted curve on a triangular grid, which is

\begin{description}
	\item [self-similar] The macroscopic behavior of the curve is the same as its microscopic behavior
	\item [self-avoiding] The curve is made from one continuous line that never intersects with itself.
	\item [edge-covering] The curve travels across every cell of its grid, i.e. it fills up a given, arbitrary area completely
	\item [plane-filling] The curve grows outward on the 2D plane without bounds. Infinitely iterated it covers the full plane.
\end{description}

\begin{figure}[h]
\centering
\begin{subfigure}{.33\textwidth}
  \centering
  \includegraphics[width=.7\linewidth]{flosnek_4it}
  \caption{4 iterates}
\end{subfigure}%
\begin{subfigure}{.33\textwidth}
  \centering
  \includegraphics[width=.7\linewidth]{flosnek_5it}
  \caption{5 iterates, rescaled}
\end{subfigure}%
\begin{subfigure}{.33\textwidth}
  \centering
  \includegraphics[width=.7\linewidth]{flosnek_8it}
  \caption{8 iterates, rescaled}
\end{subfigure}
\caption{Rendering of \textsc{Gosper's flowsnake}, with \gls{bounding box}}
	\label{fig:flosnek}
\end{figure}

The curve is obtained by iterating a so-called \gls{lsys}.

Originally introduced by biologist \textsc{Aristid Lindenmayer} in 1968 "as a foundation for an axiomatic theory of development"\citep[Preface]{Prusinkiewicz2013}. Originally intended to describe the growth of plants, it was found to be a useful language for a wide class of fractal geometries.
An \gls{lsys} consists of an initial string - the \textit{axiom} - which is manipulated via a set of substitution \textit{rules}, the result of which is a called the first \textit{iterate} of the \gls{lsys}. Subsequent \textit{iterates} are obtained by applying the \textit{rules} to the current \textit{iterate}.

The \gls{lsys} for figure \ref{fig:flosnek} is given in \citet[p.7]{Arndt2016} to be 
\begin{quote}
	\centering
	L(\textit{axiom}) \quad L $\rightarrow$ L+R++R-L- -LL-R+ \quad and \quad R $\rightarrow$ -L+RR++R+L- -L-R
\end{quote}

It yields the following \textit{iterates}:\\
\begin{table}[htb]
\begin{tabular}{r|l}\hline
\textit{axiom}& L\\\hline
First iterate 	& L+R++R-L--LL-R+\\\hline
Second iterate 	& L+R++R-L--LL-R++-L+RR++R+L--L-R++-L+\\
		& RR++R+L--L-R-L+R++R-L--LL-R+--L+R++R-L\\
		& --LL-R+L+R++R-L--LL-R+--L+RR++R+L--L-R+\\\hline
\end{tabular}
\caption{\textsc{Gosper's flowsnake} iterates 0-2}
\label{desc:lsysit}
\end{table}

The characters in this string are assigned special meaning with respect to the curve:
\begin{description}
	\item[Alphabetic character] Denotes drawing a line from the current position forward by a defined length. Forward being defined as the current direction
	\item[+ -] Characters denoting changes in direction by a set amount, e.g. in figure \ref{fig:flosnek} by $\pm60^\circ$, creating a triangular grid of movement
\end{description}

Those rules define a way to render a curve given by an \textit{iterate} of a \gls{lsys}, and can be extended by other, auxiliary characters, e.g. \_ which changes color of subsequent segments.

\begin{figure}[hb]
	\centering
	\includegraphics[width=0.5\textwidth]{flosnek_color}
	\caption{3-iterate rendering of \textsc{Gosper's flowsnake} with axiom \textrm{L\_L} and a rounding factor of 0.5}
\end{figure}

\section{Problem Statement}
In 2016, Prof. Jörg Arndt, the supervisor of this thesis, conducted research on finding plane-filling curves for \gls{lsys} with one non-constant character \citep{Arndt2016} and presented 2D renderings of the curves found.

The tools used to get from a \gls{lsys} description to a graphical, pdf-embeddable rendering of the iterated curve were assortments of chained commandline scripts, leaving much to be desired in flexibility and ease-of-use.

Thus, a request was issued for the creation of a \emph{cross-platform, maintainable and extensible} software that is able to create, visualize and export a \gls{pfc} from its \gls{lsys} description.

\section{Approach}
Since the additional requirements introduce significant complexity on top of the "implement a renderer/exporter" core task, a structured approach to finding a solution is taken using decomposition methods from software engineering according to the following procedure.
\begin{enumerate}
\item Requirements to a solution are formulated and discussed.
\item Available technologies for satisfying given requirements are researched.
\item An architectural model is created using \gls{uml} to define the system architecture.
\item The model is implemented and refined/amended to address issues arising during the implementation process.
\item The software is tested during development with \gls{ci} and \gls{unit test} techniques.
\item A short investigation of application performance is conducted.
\end{enumerate}

\section{Requirements to a Solution}
In order for the tool to be useful in research, requirements to a satisfactory solution were discussed with and agreed upon by Prof. Arndt, which are given in figure \ref{fig:directreq}. Two possible workflows are shown in the use-case diagram \ref{fig:uc}.

\begin{figure}[htb]
	\centering
	\includegraphics[width=0.8\textwidth]{DirectRequirements}
	\caption{Direct user requirements to a solution}
	\label{fig:directreq}
\end{figure}


\begin{figure}[htb]
	\centering
	\includegraphics[width=0.8\textwidth]{UseCaseDiagram}
	\caption{Workflow with the pfcrender tool}
	\label{fig:uc}
\end{figure}

These requirements are discussed in detail in the following sections and an architecture is formulated that satisfies them.
