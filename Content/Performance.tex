\chapter{Application Performance}

\section{General Optimization}
\subsubsection{Runtime Optimization}
When optimizing a program for performance, it is essential to know where in the code the program spends the most time during typical execution, as optimizing 80\% of usecases by 20\% will be much more noticeable by users than speeding up 20\% of usecases by 80\% and is thus a better target for engineering effort.

While measuring the overall execution time of a program is as simple as starting and stopping a timer, looking for timings inside of executable, i.e. at the function level, needs additional instrumentation of the code that is executing. Tools that provide this instrumentation are called application profilers.

%TODO: WHAT DO PROFILERS DO EXACTLY?

One such tool - callgrind - is part of the valgrind analysis suite and well integrated into QtCreator.

\subsubsection{Compiler Optimizations and Inlining}
%TODO: Define stack frame
If a method is simple, but executes a billion times, the cycles spent saving and restoring stack frames while jumping between functions will make up a significant portion of the execution time of the program.

This is where so called inlining can help improve the runtime performance - as with all optimizations - at a cost. The tradeoff in this case being executable size.

Inlining a function means copying the content of its body to the place where it is executed. This skips stack frame saving and a jump/return instruction pair, but (TODO: show assembler of inlined for-loop) leads to the function body being duplicated, i.e. increasing code size with each execution.

While compilers do this type of optimization automatically, a keyword inline exists, that serves as a compiler hint and actually makes gcc/clang more likely to inline a function. Care has to be taken though, since inline has a different meaning when used on a function declaration. (in this case setting the function's definition as local to this translation unit, i.e. other TUs that include the header need to provide their own, identical function definition.

Since - contrary to embedded systems - program space and RAM are usually not a bottleneck on desktop systems, it often makes sense to optimize a program for runtime performance and ignore the larger executable size.

Modern compilers are able to optimize a program by several methods automatically, a user just needs to tell them how much of a tradeoff he is willing to take. The configuration (e.g. on GCC) for this feature is a commandline switch called optimization level. This can be -OS (optimize for size), nothing or -O0 (dont optimize) or -O1 through -O3, optimizing for performance with increasing aggressiveness / tolerance for increses in codesize.

The standard mode for release software is -O2.

\subsubsection{Link-Time Optimization}
Compilers can only optimize on the code they know, i.e. the code in the current translation unit.

Some optimizations (TODO: Which) only become possible when the full program is known, which is link-time. LTO is thus also known as full-program-optimization.

Insead of optimizing on a TU base, it optimizes llvm bytecode (TODO: llvm erkären?)

-flto=full

\section{Timing Measurements}

