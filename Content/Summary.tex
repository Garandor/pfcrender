\chapter{Summary}
\label{cha:Summary}

The main driver in the decision to use the Qt framework was the decision to introduce a plugin architecture, because of the QLibrary abstraction they offer. Qt is not the most lightweight framework though, so if runtime extensibility is sacrificed, it would be possible to go with one of the much leaner frameworks like \ref{sfml} or \ref{cairo}, and possibly gain application performance.

\section{Followups}
\label{sec:Followups}
A followup on this thesis could be integration of model rendering to a OpenGL framebuffer object using a 3rd-party graphics library (e.g. \ref{cairo} or \ref{sdl}) and rendering that as a QQuickFramebufferObject\furl{http://doc.qt.io/qt-5/qquickframebufferobject.html} - thus evaluating performance of more drawing engines than the Qt-native QPainter.
