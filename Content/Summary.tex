\chapter{Summary}
\label{cha:Summary}
This work originated from the need for an \emph{extensible, maintainable, cross-platform} solution for rendering \gls{pfc}s to facilitate ongoing research in the field.
These criteria span a wide range of software engineering techniques. The most accessible to small-scale open-source projects of these were selected for implementation of the architecture.
The result is a \emph{self-documenting, self-testing and self-deploying open-source project that encourages collaboration}.

The software architecture itself was designed specifically for extensibility by isolating non-core functionality to plugins and holding configuration information in a central, automatically filled repository and guarantees cross-platform operation by building on top of the Qt Framework.

The resulting implementation is a solution to most requirements given in fig \ref{fig:directreq} by design, the required rendering performance of 10 frames per second was verified via benchmarking.


At the time of writing, SVG (section \ref{sec:svg}) and PDF (section \ref{sec:pdf}) plugins are in a proof-of-concept state and a possibility of configuration via \gls{gui} has not been implemented. Along with extending the application, e.g. by implementing a \term{PFC library tool} like proposed in section \ref{sec:archplug}, refining \gls{gui} functionality and increasing the configurability of the export plugins are recommended improvements to the existing solution.

Rendering performance of the application currently is limited through the one-shot characteristic of model creation, as no in-program facility of generating a new \gls{pfc} yet exists apart from specification on the \gls{cli}, resulting in rendering of multiple \gls{pfc}s requires application restarts. This incurrs the startup time delay of plugin loading each time. A \gls{gui} embedded configuration screen for the import plugin or a facility for rerunning imports from within the application would thus be a desirable extension.
