\chapter{Introduction}
\label{cha:Intro}

\section{Motivation}

So-called plane-filling-curves are an area of research of interest to many different industries, ranging from toolpath generation for 3D-Printers to fractal art.

Prof. Arndt is an active researcher in this field, and issued a request for a system-portable and extensible software for visualizing and exporting the graphical representations of those curves.

Hier sieht man den Unterschied zwischen z.B. und \zB wenn man auf den Leerraum achtet.

\section{Ziel der Arbeit}
Lorem ipsum dolor sit amet, consectetuer adipiscing elit. Nulla ac ipsum a metus viverra tempor. Nunc sem. Nulla nec urna eu nibh vehicula convallis. Integer ac turpis. Donec mauris enim, dignissim quis, scelerisque ac, rhoncus id, sapien. Donec turpis felis, cursus in, varius vitae, mollis ac, lorem. Integer a dui sit amet eros nonummy aliquet. Donec egestas adipiscing tellus. Nulla iaculis. Aliquam erat volutpat. Curabitur posuere, eros vitae accumsan semper, risus erat viverra erat, eu vehicula mi leo at elit. Fusce luctus. Fusce vehicula pretium diam. Nunc sed arcu ut erat suscipit fermentum.


\section{Approach}
The above mentioned problem is complex enough to consider a structured approach to finding a solution using decomposition methods from software engineering.

First, the requirements were formulated and discussed with the client.

Then, research was conducted on available technologies for satisfying given requirements.

After technologies were agreed upon, a \glossary{UML}{Unified Markup Language - A systems modeling standard by the Object Management Group} model of the basic system was designed to define interfaces and the basic structure of the application.

This model was then implemented and refined/amended to address issues arising during the implementation process and tested during development with \glossary{CI}{Continuous Integration} and \glossary{Unit Testing}{} techniques.

A short investigation of application performance was then conducted.

\section{Requirements to the Architecture}

%\figure{Requirement Matrix}
